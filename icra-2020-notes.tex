\documentclass[12pt]{article}

\usepackage{setspace}
\usepackage{amssymb, amsfonts, amsmath, mathrsfs, color, fancyhdr, tikz-cd, adjustbox, bbm, xcolor, wasysym}
\usepackage[thmmarks]{ntheorem}
\usepackage[ntheorem,framemethod=TikZ]{mdframed}
\usepackage{diagbox}
%Disabling for now to speed up compilation
\usepackage{hyperref}
\hypersetup{
	colorlinks = true,
	linkcolor = [rgb]{0,0,0.5},
	citecolor = [rgb]{0.6,0,0},
	urlcolor = [rgb]{0,0,0.5}
}

% Suppress mdframed telling us about "bad breaks". I can already see them.
\usepackage{silence}
\WarningFilter{mdframed}{You got a bad break}
\makeatletter
\mdf@PackageWarning{You got a bad break\MessageBreak
  because the last split box is empty\MessageBreak
  You have to change the settings}
\makeatother

\usepackage[style=alphabetic, bibencoding=utf8]{biblatex}
%Set the bibliography file
\bibliography{sources}

%Document-Specific includes
\usepackage{ytableau}

%Replacement for the old geometry package
\usepackage{fullpage}

%Input my definitions
\input{mydefs.tex}

%%%%%%%%%%%%%%%%%%%%%%% Customize Below %%%%%%%%%%%%%%%%%%%%%%%%%%%%%%
%%%%%%%%%%%%%%%%%%%%%%%%%%%%%%%%%%%%%%%%%%%%%%%%%%%%%%%%%%%%%%%%%%%%%%

%header stuff
\setlength{\headsep}{24pt}  % space between header and text
\pagestyle{fancy}     % set pagestyle for document
\lhead{Research Discussion Notes} % put text in header (left side)
\rhead{Nico Courts} % put text in header (right side)
\cfoot{\itshape p. \thepage}
\setlength{\headheight}{15pt}
%\allowdisplaybreaks

% Document-Specific Macros
\DeclareMathOperator{\Spc}{Spc}
\DeclareMathOperator{\Pol}{Pol}
\newcommand{\vi}{\mathbf{i}}
\newcommand{\vj}{\mathbf{j}}
\newcommand{\vk}{\mathbf{k}}
\newcommand{\vl}{\mathbf{l}}
\newcommand{\vm}{\mathbf{m}}
\DeclareMathOperator{\rad}{rad}
\DeclareMathOperator{\ev}{ev}
\DeclareMathOperator{\add}{\mathbf{add}}

%make the title page
\title{ICRA 2020 Notes}
\author{Typeset by: Nico Courts}
\date{November 2020}

\begin{document}

\maketitle
\begin{abstract}
    What follows are my notes from the lectures and workshops I attended during the 2020 (online) International Conference on the Representation of Algebras. None of what follows is my own work besides the typing and any errors in transcription are my own.
    
    Workshops are three-lecture series that I will include all under the same section. There are also one-off lectures that will be in their own section.
\end{abstract}
\section{Pavel Etingof: Symmetric Tensor Categories}
\subsection{Part I: Algebras and Representation Theory without Vector Spaces}
The motivating idea here is that symmetric tensor categories are the correct generalization of group representations for more interesting objects of study. Throughout we let $G$ be a group, $k$ an algebraically-closed field, and we consider the category $\Rep_k(G)$ of (finite dimensional) representations of $G$ over $k$.

Recall that if $X,Y\in\Rep(G)$, $\Hom(X,Y)$ is the set of ``intertwining operators'' (satisfying $G$ linearity). What kind of structures and properties does this category have?
\begin{itemize}
    \item It is $k$-linear and abelian (in particular we can take kernels, cokernels, and images of morphisms, direct sums, etc)
    \item It is artinian (locally finite); that is, objects have finite length (composition series) and $\Hom(X,Y)$ are finite dimensional over $k$.
    \item It is monoidal (has associator isomorphisms for every triple, etc).
    \item It is rigid (we have a contravariant involutive endofunctor $X\to X^\ast$) such that $X^{\ast\ast}$ is isomorphic to $X$ and furthermore we have evaluation and coevaluation maps
    \[\operatorname{ev}_X:X^\ast\otimes X\to\mathbbm{1}\qquad \operatorname{coev}_X:\mathbbm{1}\to X\otimes X^\ast\]
    satisfying some properties.
    \item The $\otimes$ of morphisms is bilinear.
    \item $\End(\mathbbm{1})=k$
\end{itemize}
\begin{defn}
A category $\calC$ with such structures and properties is called a symmetric tensor category (STC). For an example, $\Rep(G)$ is an STC.
\end{defn}

A natural question to ask is \textit{are all STC of this form?} The answer to this is no. We can take $G$ to be an algebraic group or more generally an affine group scheme. THen consider $\calC=\Rep G$. This means that we have a (commutative) Hopf algebra $H$ that is the ring of functions on $G=\Spec H$. Then representations of $G$ are exactly finite dimensional comodules over $G$. This generalizes the previous setting: if $G$ is an abstract group, one can define the (reduced) affine group scheme $\widehat G$ called the \emph{proalgebraic completion} of $G$. This is defined to be the group scheme such that
\[\Rep \hat G\cong \Rep_k(G).\]

Then $\calO(\hat G)$ are the matrix elements of finite dimensional reps of $G$, or equivalently the functions on $G$ of the form $\varphi(g)=(f, gv)$ for $v\in V$ and $f\in V^\ast$.

\textbf{Question:} Did this show what we wanted to see?

Is that it? Still no. The counterexample this time comes from supervector spaces. Here we require $\ch k\ne 2$ and we get a $\bbZ/2$-graded vector spaces $V=V_0\oplus V_1$ which is symmetric except for the Koszul sign rule:
\[x\otimes y\cong (-1)^{|x||y|}y\otimes x.\]
Then it is not of the above form.

As an example, if $k=\bbC$, we can define the dimension of an object as the composition
\[\mathbbm{1}\xrightarrow{coev_X}X\otimes X^\ast\xrightarrow{\mathrm{twist}}X^\ast\otimes X\xrightarrow{ev_X} k\]
In the case of a super vector space, we can do the same thing to get the dimension to be $\dim X_0-\dim X_1$ which gives us the Euler characteristic (in particular it could be negative).

But in $\Rep G$, the dimensions are nonnegative. We can generalize previous ideas further: affine supergroup schemes are $H=H_0\otimes H_1$ that are supercommutative Hopf superalgebras (commutative Hopf algebra objects in supervect). Here $G=\Spec H$ is an affine supergroup scheme and $\Rep G$ are the finite dimensions $H$-comodules. This is not quite the furthest generalization: we can further take $z\in G(k)$ such that $z^2=1$ and then $z$ action on $\calO(G)=H$ by parity:
\[z|_{H_0}=1,z|_{H_1}=-1\]
and $\Rep(G,z)$ is the representation category of $G$ on super vector spaces on which $z$ acts by parity.

Surely we are done now, right? Not quite. STC can in general be very big. Let $X\in\calC$ and write $l_n(X)$ for the length of $X^{\otimes n}$. If $\calC=\Rep(G)$
or $\Rep(G,z),$ then $l_n(X)$ grows at most exponentially with $n$, since 
\[l_n(X)\le (\dim_k X)^n.\]
However, it turns out that there are categories where this growth is more than exponential. These are called \emph{Deline categories} which are defined as interpolations $\Rep S_t$, for instance $\Rep \GL_t$ or $\Rep \operatorname{Sp}_t$ for some $t\in \bbC$. 

\begin{defn}
We say $\calC$ is of \emph{moderate growth} if $l_n(X)$ grows at most exponentially for all $X$.
\end{defn}
Refining our question: is any STC of moderate growth of the form $\Rep(G,z)$ for some $G$ and $z$? The answer:
\begin{thm}[Deligne 2002]
Every symmetric tensor category over $k$ of characteristic zero of moderate groth is of the form $\Rep(G,z).$
\end{thm}
\begin{rmk}
The positive characteristic case is false.
\end{rmk}

In fact, given a STC, we can construct $G$ or $(G,z)$ using \textbf{Tannakian formalism}. Let $\calC=\Rep G$ and let $\calF:\Rep G\to \Vectk$ be the forgetful functor.
\begin{thm}
$\calF$ is the unique exact faithful functor preserving $\otimes$ (that is monoidal).
\end{thm}
\begin{thm}
The group giving us our representation category is $G=\Aut_\otimes(\calF)$, the group of automorphisms of $F$ as a monoidal functor.
\end{thm}
This functor is called a \emph{fiber functor} and for $\calC=\Rep(G,z)$ we have a unique superfiber functor such that $G=\Aut_\otimes(\calF)$ and $z$ is the parity automorphism.

We say that $\Rep G$ is \textbf{Tannakian} and $\Rep(G,z)$ is \textbf{supertannakian}. There was also a result of Deligne at the end that I missed but I bet we can coerce the speaker to repeat it next time. :)

\section{Srikanth Iyengar: Duality for Gorenstein Algebras}
\subsection{Part I: Gorenstein Algebras}
The slogan here is that we are guided, but not led by commutative algebra.

\subsubsection{Commutative Gorenstein Rings}
Throughout this (sub) section we will let $R$ be a commutative Noetherian ring. Examples here comes from coordinate rings of algebraic varieties. If we say $(R,\frakm, k)$ is local, it means the usual thing. We say that $R$ is \emph{Gorenstein} if $\injdim_R R<\infty$.

If $M\in \lmod R$, then $\injdim_R M<\infty$ if and only if $\Ext_R^i(-,M)=0$ for all $i>>0$ iff $\Ext_R^i(k,M)=0$ for all i large enough iff $\Ext_R^i(k,M)=0$ for some $i\ge \dim R+1$ where $\dim$ denotes Krull dimension. Thus our final characterization is that $\Ext_R^{d+1}(k,M)=0.$
This condition is easy to check by using a projective resolution of $k$! This is a result of Ischebeck.

In summary,
\[\injdim_R M<\infty \Leftrightarrow \injdim_R(M)\le \dim R\]
and in fact (although it is nontrivial) it is equal to $\dim R!$ This is a result of Roberts in the ``new intersection theorem.'' THus $R$ is Gorenstein if and only if $\injdim R=\dim R$. In fact (when $R$ is local?), we get 
\[\Ext_R^i(k,R)=\left\{\begin{matrix}
0 & i\ne\dim R,\\
k & i=\dim R
\end{matrix}\right.\]
that is, $R\Hom(k,R)\simeq \Sigma k$ in $\D(R).$

Now if $R$ is any commutative Noetherian ring, $R$ is Gorenstein if $R_\frakp$ is Gorenstein for all $\frakp\in\Spec R$ (suffices to check on maximals).
In this case, $\injdim R=\dim R$ Thus $R$ is Artinian and Gorenstein iff $\injdim R=0$ iff $R$ is self-injective.

It can happen that $\dim R=0$. According to Nagata we can construct a commutative noetherian ring $R$ such that $\gldim R_\frakp<\infty$ ($R$ is regular) and yet $\dim R=\infty.$ So Gorenstein rings are not automatically Iwanaga-Gorenstein.

Some examples: 
\begin{itemize}
    \item Regular rings
    \item Determinantal rings: let $X=(X_{ij})$ be an $n\times n$ matrix of variables and $k[X]$ the polynomial ring over the $x_{ij}.$ For $1\le t\le n$, consider $I_t(X)$, the ideal generated by the $t\times t$ minors of $X$. Then $R=k[X]/I_t(X)$ is Gorenstein (it is picking up the matrices of rank larger than $t$. An example is $k[x,y,u,v]/(xy-uv)$
    \item Rings of invariants: if $k$ is a field, and $G\le \operatorname{SL}(k^n)$ is a finite subgroup then $G$ acts on $S\eqdef k[x_1,\dots,x_n]$ and $R=S^G$ is Gorenstein (due to Watanabe).
\end{itemize}
\subsubsection{Gorenstein Algebras}
Let $A$ be an associative ring. Then $A$ is a \emph{Noether $R$ algebra} if $R$ is commutative Noetherian, $A$ is an $R$ module with $\eta:R\to A$ such that $\Im(\eta)\subseteq Z(A)$ and finally that $A$ is a finitely-generated $R$-module (which implies $A$ is Noetherian on both sides).

$A$ is a \emph{Gorenstein $R$-algebra} if also $A$ is projective as an $R$-module and the functor $R\hookrightarrow \End_R(A)$ is faithful and furthermore that $A_\frakp$ Is Iwanaga-Gorenstein (IG) for all $\frakp\in\Spec R$ and $\injdim A_\frakp<\infty$ (on both sides. Note that $A$ is a Gorenstein $R$-algebra if and only if $A\op$ is as well.

Some examples include commutative Gorenstein rings themselves and finite dimensional algebras over $k$ that is Gorenstein. The group algebra RG for $G$ a finite group (scheme) are Gorenstein $R$-algebras. Furthermore $\Lambda(R^n)$ is Gorenstein. We also have $M_n(R)$ are Gorenstein. Finally infinite-dimensional gentle algebras are finite dimensional and central over a polynomial ring in 1 variable ($k[x]\hookrightarrow A$). Then $A$ is ``Gorenstein'' except $A$ might not be projective over $k[x]$, although there are many interesting examples that are.

One take: the definition of Gorenstein $R$-algebras ought to be relaxed to allow $\projdim_A R<\infty$ (rather than being projective). This is a suggestion that makes things easier, although we will be working with the usual definition in what follows. Buchweitz did consider this idea in his notes.

When $R$ is commutative Noetherian and $M\in \lmod R$, the projective dimension over $R$ of $M$ is finite iff $\projdim_{R_\frakp}M_\frakp$ is finite for all $\frakp.$ This is \textbf{not true} in general for $\injdim.$

\subsubsection{Representation Theory}
Let $(R,\frakm, k)$ be local, $M\in\lmod R$. Then $M$ is \emph{Maximum Cohen-MacCaulay (MCM)}
 if $\depth_R M=\dim R$ (or $M=0$). This is equivalent to $\Ext_R^i(k,M)=0$ for $i<\dim R$ iff $\Ext_R^i(M,R)=0$ for some $i$ (sorry I he scrolled up too fast).
 
 Now for any $M\in \lmod R$, $\Ext_R^i(M,R)=0$ for all $i>\dim R + 1$ so the $i^{th}$ syzygy module $\Omega^i_R M$ is MCM for all $i\ge \dim R$. 
 
 \textit{I missed a whole blurb here.}
  
 It is a result due to Goto that $M\in \lmod R$ implies that $\Ext_R^i(M,R)=0$ for all $i$ large enough (no bound on $i$ apriori). This means the functor
 \[R\Hom_R(_,R):\Db(\lmod R)\op\xrightarrow{\sim}\Db(\lmod R)\]
 is an equivalence, which is another form of local duality per Grothendieck.
 
 \subsubsection{Noncommutative case}
 If $A$ is a Gorenstein $R$-algebra, $M$ is MCM is $\Ext_A^i(M,A)=0$ for all $i\ge 1.$ Note that this doesn't depend on $R$. ALso note that if $R\hookrightarrow A$ then if $A$ is Gorenstein, this implies $R$ is (the converse is not true). Furthermore, if $M\in\lmod A$ and $M$ is MCM over $A$, it is also MCM over $R$ (again, the converse is not true).
 
 The above examples points to a difference between the commutative and noncommutative cases: if $R\hookrightarrow S$ is an inclusion of commutative rings and if $S$ is finite dimensional over $R$, $M$ is MCM over $S$ iff it is over $R$.
 
 \begin{thm}[Iyengar, Krause]
 For any $M\in \lmod A$, $\Ext_A^i(M,A)=0$ for all $i>>0$. Thus $\Omega_A^iM$ are MCM for all $i>>0.$
 \end{thm}
 In fact,
 \[R\Hom(_,A):\Db(\lmod A)\to \Db(\lmod{A\op})\]
 is an equivalence.
 
 A problem: how does one check if $A$ is a Gorenstein $R$-algebra? If we have $(R,\frakm, k)$, one can check $\Ext_R^{d+1}(k,R)=0$ which can be computed by writing down a free resolution of $k$. What about general $R$ that are commutative noetherian? 
 
 It is known that $R$ is Gorenstein iff $R$ is Cohen-MacCaulay (can be computed) and it satisfies a condition on $\Ext$ which is not necessarily computable. There is a conjecture due to \textbf{FIND NAME} that would make it computable, however.
\end{document}
